\documentclass{article}
\usepackage{amsmath}
\usepackage{amsfonts}

\begin{document}
	\title{Filter Notes}
	\author{Stewart Nash}

	\maketitle
	\section{Transmission Lines}
		\subsection{Group Velocity}
In a source-free, linear, isotropic, homogeneous region, Maxwell's curl equations in phasor form are
\begin{align}
	\nabla\times\vec{E}&=-j\omega\mu\vec{H},\\
	\nabla\times\vec{H}&=j\omega\epsilon\vec{E}.
\end{align}
These yield the wave equations which are known as the Helmholtz equations
\begin{align}
	\nabla^2\vec{E}+\omega^2\mu\epsilon\vec{E}&=0,\\
	\nabla^2\vec{H}+\omega^2\mu\epsilon\vec{H}&=0.
\end{align}
The constant $k=\omega\sqrt{\mu\epsilon}$ is known as the \emph{propagation constant}, phase constant, or wave number. It is given in units of inverse length.
We could obtain a plane-wave solution of the Helmholtz equation in an arbitrary direction. But for generality, let us assume an arbitrary direction $r$ which could represent a radial component of a spherical solution to the Helmholtz equation. The velocity of a wave is called the phase velocity because it is the velocity at which a fixed phase point on the wave travels. It is given by
\begin{equation}
	v_p=\frac{dr}{dt}=\frac{\omega}{k}=\frac{1}{\sqrt{\mu\epsilon}}.
\end{equation}
In a lossy medium with conductivity $\sigma$ Maxwell's curl equations are
\begin{align}
	\nabla\times\vec{E}&=-j\omega\mu\vec{H},\\
	\nabla\times\vec{H}&=j\omega\epsilon\vec{E}+\sigma\vec{E}.
\end{align}
The resulting Helmholtz wave equations are
\begin{align}
	\nabla^2\vec{E}+\omega^2\mu\epsilon\left(1-j\frac{\sigma}{\omega\epsilon}\right)\vec{E}&=0,\\
	\nabla^2\vec{H}++\omega^2\mu\epsilon\left(1-j\frac{\sigma}{\omega\epsilon}\right)\vec{H}&=0.
\end{align}
If we define the \emph{complex propagation constant} as
\begin{equation}
	\gamma=\alpha+j\beta=j\omega\sqrt{(\mu\epsilon)\left(1-j\frac{\sigma}{\omega\epsilon}\right)}
\end{equation}
with attenuation constant $\alpha$ and phase constant $\beta$. In this case, we have a phase velocity given by
\begin{equation}
	v_p=\frac{dr}{dt}=\frac{\omega}{\beta}.
\end{equation}
The telegrapher's equations yield the phasor form of wave propagation of voltage and current on a generalized transmission line in the direction $\mathbf{r}$
\begin{align}
	D^2_{\hat{\mathbf{r}}}\tilde{V}(\mathbf{r})-\gamma^2(\omega)\tilde{V}(\mathbf{r})&=0,\\
	D^2_{\hat{\mathbf{r}}}\tilde{I}(\mathbf{r})-\gamma^2(\omega)\tilde{I}(\mathbf{r})&=0.
\end{align}
On a lossless line, the propagation constant and phase constant are
\begin{align}
	\gamma&=j\omega\sqrt{LC},\\
	\beta&=\omega\sqrt{LC}.
\end{align}
The phase velocity on a lossless line is
\begin{equation}
	v_p=\frac{\omega}{\beta}=\frac{1}{\sqrt{LC}}.
\end{equation}
On a lossy line, the propagation constant and phase velocity are
\begin{align}
	\gamma&=\alpha+j\beta=\sqrt{(R+j\omega{L})(G+j\omega{C})},\\
	v_p&=\frac{\omega}{\beta(\omega)}.
\end{align}
The group velocity can be interpreted physically as the velocity at which a narrowband signal propagates. Consider a system whose transfer function is given by $H(\omega)=Ae^{-j\beta{r}}$ and whose input signal $x(t)$ is a baseband function $u(t)$ modulated by the tone $\mathrm{Re}\{e^{j\omega_0t}\}$ so that $X(\omega)=U(\omega-\omega_0)$. In the time domain, the output signal is given by
\begin{equation}
	y(t)=\frac{1}{2\pi}\mathrm{Re}\left\{\int^{\infty}_{-\infty}{AU(\omega-\omega_0)e^{j(\omega{t}-\beta{r})}\,d\omega}\right\}.
\end{equation}
If $U(\omega)$ is narrowband---$\omega_m\ll\omega_0$ where $\omega_m$ is the maximum frequency of the baseband function $u(t)$---then the propagation constant $\beta$ can be linearized using a Taylor series expansion about $\omega_0$,
\begin{equation}
	\begin{split}
		\beta(\omega)&\approx\beta(\omega_0)+\left.\frac{d\beta}{d\omega}\right|_{\omega=\omega_0}(\omega-\omega_0)\\
		&=\beta_0+{\beta_0}^\prime(\omega-\omega_0).
	\end{split}
\end{equation}
This means that the output signal is
\begin{equation}
	y(t)\approx{Au(t-{\beta_0}^\prime{r})\cos{(\omega_0t-\beta_0r)}}
\end{equation}
and the group velocity is
\begin{equation}
	v_g\approx\frac{1}{{\beta_0}^\prime}=\left.\left(\frac{d\beta}{d\omega}\right)^{-1}\right|_{\omega=\omega_0}.
\end{equation}
		\subsection{Group Delay}
Group delay and phase delay describe the differential delay times of a signals frequency components as they propagate through an LTI system. Consider a system whose transfer function is given by $H(\omega)=|H(\omega)|e^{j\phi(\omega)}=Ae^{j\phi(\omega)}$ and whose input signal $x(t)$ is a baseband function $u(t)$ modulated by the tone $\mathrm{Re}\{e^{j\omega_0t}\}$ so that $X(\omega)=U(\omega-\omega_0)$. In the time domain, the output signal is given by
\begin{equation}
	y(t)=\frac{1}{2\pi}\mathrm{Re}\left\{\int^{\infty}_{-\infty}{AU(\omega-\omega_0)e^{j(\omega{t}-\phi(\omega))}\,d\omega}\right\}.
\end{equation}
If $U(\omega)$ is narrowband---$\omega_m\ll\omega_0$ where $\omega_m$ is the maximum frequency of the baseband function $u(t)$---then the phase response $\phi$ can be linearized using a (first-order) Taylor series expansion about $\omega_0$,
\begin{equation}
	\begin{split}
		\phi(\omega)&\approx\phi(\omega_0)+\left.\frac{d\phi}{d\omega}\right|_{\omega=\omega_0}(\omega-\omega_0)\\
		&=\phi_0+{\phi_0}^\prime(\omega-\omega_0).
	\end{split}
\end{equation}
We then make a change of variables $\xi=\omega-\omega_0$ to obtain
\begin{equation}
	\begin{split}
		y(t)&\approx\frac{1}{2\pi}\mathrm{Re}\left\{\int^{\infty}_{-\infty}{AU(\omega-\omega_0)e^{j(\omega{t}-\phi_0-{\phi_0}^\prime(\omega-\omega_0))}\,d\omega}\right\}\\
		&=\frac{A}{2\pi}\mathrm{Re}\left\{e^{j(\omega_0t-\phi_0)}\int^{\infty}_{-\infty}{AU(\xi)e^{j(t-{\phi_0}^\prime\xi)}\,d\xi}\right\}\\
		&=A\mathrm{Re}\left\{u(t-{\phi_0}^\prime)e^{j(\omega_0t-\phi_0)}\right\}\\
		&=Au(t-{\phi_0}^\prime)\cos{(\omega_0t-\phi_0)}.
	\end{split}
\end{equation}
Given that the group delay is defined as
\begin{equation}
	\tau_g\equiv-\left.\frac{d\phi}{d\omega}\right|_{\omega=\omega_0}
\end{equation}
our output signal is
\begin{equation}
	y(t)\approx{Au(t-\tau_g)\cos{(\omega_0t-\phi_0)}}.
\end{equation}
The phase delay is defined as
\begin{equation}
	\tau_\phi\equiv-\frac{\phi}{\omega}.
\end{equation}
		\subsection{Conclusion}
If we have a medium of length $l$ with propagation constant $\gamma=\alpha+j\beta$ then the system phase is $\phi=-\beta{l}$ so that
\begin{equation}
	\begin{split}
		\tau_g(\omega)&=-\frac{d\phi}{d\omega}\\
		&=l\frac{d\beta}{d\omega}\\
		&=\frac{l}{v_g}.
	\end{split}
\end{equation}

	\section{Transmission Matrix}
The transition matrix or \emph{ABCD} matrix allows cascading of network connections, and is thus a popular choice for representing networks in network analysis. The \emph{ABCD} matrix for a two-port network is defined as follows:
\begin{equation}
	\begin{bmatrix}
		V_1\\
		I_1
	\end{bmatrix}=
	\begin{bmatrix}
		A&B\\
		C&D
	\end{bmatrix}
	\begin{bmatrix}
		V_2\\
		I_2
	\end{bmatrix}.
\end{equation}
In general, we may have the transition matrix between ports $i$ and $i+1$:
\begin{equation}
	\begin{bmatrix}
		V_i\\
		I_i
	\end{bmatrix}=
	\begin{bmatrix}
		A_i&B_i\\
		C_i&D_i
	\end{bmatrix}
	\begin{bmatrix}
		V_{i+1}\\
		I_{i+1}
	\end{bmatrix}.
\end{equation}
And we may cascade the transition matrices between ports $i$, $i+1$ and $i+2$:
\begin{equation}
	\begin{bmatrix}
		V_i\\
		I_i
	\end{bmatrix}=
	\begin{bmatrix}
		A_i&B_i\\
		C_i&D_i
	\end{bmatrix}
	\begin{bmatrix}
		A_{i+1}&B_{i+1}\\
		C_{i+1}&D_{i+1}
	\end{bmatrix}	
	\begin{bmatrix}
		V_{i+2}\\
		I_{i+2}
	\end{bmatrix}.
\end{equation}

	\section{Impedance Matching}
As long as the load impedance has a positive real part, a matching network can be found.
		\subsection{\emph{L} Network}
The \emph{L}-section can be used as a matching network. It comes in two configurations: (1) a series reactive element X followed by a shunt reactive element B or (2) a shunt reactive element B followed by a series reactive element X. The reactive elements may be either inductors or capacitors. Configuration 1 is used the normalized load impedance is inside the $1+jx$ circle on the Smith chart and configuration 2 should be used if the normalized load impedance is outside of this circle. The element values of the matching network may be solved either algebraically or with the use of a Smith chart.
		\subsection{Stub Tuning}
A single stub either in series or in shunt (configuration) may be used to tune. A double-stub tuner circuit may also be used.

\end{document}

