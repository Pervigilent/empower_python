\documentclass{article}
\usepackage{amsmath}
\usepackage{amsfonts}

\begin{document}
	\title{Filter Notes}
	\author{Stewart Nash}

	\maketitle
	\section{Group Delay}
In a source-free, linear, isotropic, homogeneous region, Maxwell's curl equations in phasor form are
\begin{align}
	\nabla\times\vec{E}&=-j\omega\mu\vec{H}\,,\\
	\nabla\times\vec{H}&=j\omega\epsilon\vec{E}\,.
\end{align}
These yield the wave equations which are known as the Helmholtz equations
\begin{align}
	\nabla^2\vec{E}+\omega^2\mu\epsilon\vec{E}&=0\,,\\
	\nabla^2\vec{H}+\omega^2\mu\epsilon\vec{H}&=0\,.
\end{align}
The constant $k=\omega\sqrt{\mu\epsilon}$ is known as the \emph{propagation constant}, phase constant, or wave number. It is given in units of inverse length.
We could obtain a plane-wave solution of the Helmholtz equation in an arbitrary direction. But for generality, let us assume an arbitrary direction $r$ which could represent a radial component of a spherical solution to the Helmholtz equation. The velocity of a wave is called the phase velocity because it is the velocity at which a fixed phase point on the wave travels. It is given by
\begin{equation}
	v_p=\frac{dr}{dt}=\frac{\omega}{k}=\frac{1}{\sqrt{\mu\epsilon}}\,.
\end{equation}

\end{document}

