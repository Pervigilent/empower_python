\documentclass[12pt]{article}

\usepackage{amsmath}
\usepackage{graphicx}
\usepackage{amsthm}
\usepackage{listings}
\usepackage{xcolor}

%\theoremstyle{definition}
%\newtheorem{example}{Example}[section]

% Custom command for editor
\newcommand{\editor}[1]{\gdef\@editor{#1}}
\newcommand{\printeditor}{%
  \par\vspace{1em}
  \textbf{Editor:} \@editor
}
\makeatletter

\title{Mie Scattering and Absorption (Version 1)}

\author{
  Christian M\"atzler\\[6pt]
  Institute of Applied Physics\\
  University of Bern, Switzerland\\
  \texttt{matzler@iap.unibe.ch}
}

\date{June 2002}

\editor{Stewart Nash}

\begin{document}

\maketitle
\printeditor

\begin{abstract}
A set of Mie functions has been developed in Python to compute the four Mie coefficients $a_n$, $b_n$, $c_n$ and $d_n$, efficiencies of extinction, scattering, backscattering and absorption, the asymmetry parameter, and the two angular scattering functions $S_1$ and $S_2$. In addition to the scattered field, also the absolute-square of the internal
field is computed and used to get the absorption efficiency in a way independent from the scattered field. This allows to test the computational accuracy. This first version of MATLAB Mie Functions is limited to homogeneous dielectric spheres without change in the magnetic permeability between the inside and outside of the particle. Required input parameters are the complex refractive index, $m=m'+ im''$, of the sphere (relative to the ambient medium) and the size parameter, $x=ka$, where $a$ is the sphere radius and $k$ the wave number in the ambient medium.
\end{abstract}

\section{Introduction}
This report is a description of Mie-Scattering and Mie-Absorption programs written in the numeric computation and visualisation software, MATLAB (Math Works, 1992), for the improvement of radiative-transfer codes, especially to account for rain and hail in the microwave range and for aerosols and clouds in the submillimeter, infrared and visible range. Excellent descriptions of Mie Scattering were given by van de Hulst (1957) and by Bohren and Huffman (1983). The present programs are related to the formalism of Bohren and Huffman (1983). In addition an extension (Section 2.5) is given to describe the radial dependence of the internal electric field of the scattering sphere and the absorption resulting from this field. Except for Section 2.5, equation numbers refer to those in Bohren and Huffman (1983), in short BH, or in case of missing equation numbers, page numbers are given. For a description of computational problems in the Mie calculations, see the notes on p. 126-129 and in Appendix A of BH.

\section{Formulas for a homogeneous sphere}
\subsection{Mie coefficients and Bessel functions}
Python function: mie\_abcd\\
The key parameters for Mie calculations are the Mie coefficients $a_n$ and $b_n$ to compute the amplitudes of the scattered field, and $c_n$ and $d_n$ for the internal field,
respectively. The computation of these parameters has been the most challenging part in Mie computations due to the involvement of spherical Bessel functions up to
high order. With MATLAB’s built-in double-precision Bessel functions, the computation of the Mie coefficients has so far worked well up to size parameters exceeding 10,000; the coefficients are given in BH on p.100:
\begin{align}
	a_n&=\frac{m^2j_n(mx)[xj_n(x)]'-\mu_1j_n(x)[mxj_n(mx)]'}{m^2j_n(mx)[xh_n^{(1)}(x)]'-\mu_1h_n^{(1)}(x)[mxj_n(mx)]'}\\
	b_n&=\frac{\mu_1j_n(mx)[xj_n(x)]'-j_n(x)[mxj_n(mx)]'}{\mu_1j_n(mx)[xh_n^{(1)}(x)]'-h_n^{(1)}(x)[mxj_n(mx)]'}
\end{align}

\end{document}

