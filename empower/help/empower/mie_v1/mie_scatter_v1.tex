\documentclass[12pt]{article}

\usepackage{amsmath}
\usepackage{graphicx}
\usepackage{amsthm}
\usepackage{listings}
\usepackage{xcolor}

%\theoremstyle{definition}
%\newtheorem{example}{Example}[section]

% Custom command for editor
\newcommand{\editor}[1]{\gdef\@editor{#1}}
\newcommand{\printeditor}{%
  \par\vspace{1em}
  \textbf{Editor:} \@editor
}
\makeatletter

\title{Mie Scattering and Absorption (Version 1)}

\author{
  Christian M\"atzler\\[6pt]
  Institute of Applied Physics\\
  University of Bern, Switzerland\\
  \texttt{matzler@iap.unibe.ch}
}

\date{June 2002}

\editor{Stewart Nash}

\begin{document}

\maketitle
\printeditor

\begin{abstract}
A set of Mie functions has been developed in Python to compute the four Mie coefficients $a_n$, $b_n$, $c_n$ and $d_n$, efficiencies of extinction, scattering, backscattering and absorption, the asymmetry parameter, and the two angular scattering functions $S_1$ and $S_2$. In addition to the scattered field, also the absolute-square of the internal
field is computed and used to get the absorption efficiency in a way independent from the scattered field. This allows to test the computational accuracy. This first version of MATLAB Mie Functions is limited to homogeneous dielectric spheres without change in the magnetic permeability between the inside and outside of the particle. Required input parameters are the complex refractive index, $m=m'+ im''$, of the sphere (relative to the ambient medium) and the size parameter, $x=ka$, where $a$ is the sphere radius and $k$ the wave number in the ambient medium.
\end{abstract}

\section{Introduction}
This report is a description of Mie-Scattering and Mie-Absorption programs written in the numeric computation and visualisation software, MATLAB (Math Works, 1992), for the improvement of radiative-transfer codes, especially to account for rain and hail in the microwave range and for aerosols and clouds in the submillimeter, infrared and visible range. Excellent descriptions of Mie Scattering were given by van de Hulst (1957) and by Bohren and Huffman (1983). The present programs are related to the formalism of Bohren and Huffman (1983). In addition an extension (Section 2.5) is given to describe the radial dependence of the internal electric field of the scattering sphere and the absorption resulting from this field. Except for Section 2.5, equation numbers refer to those in Bohren and Huffman (1983), in short BH, or in case of missing equation numbers, page numbers are given. For a description of computational problems in the Mie calculations, see the notes on p. 126-129 and in Appendix A of BH.

\section{Formulas for a homogeneous sphere}
\subsection{Mie coefficients and Bessel functions}
Python function: mie\_abcd\\
The key parameters for Mie calculations are the Mie coefficients $a_n$ and $b_n$ to compute the amplitudes of the scattered field, and $c_n$ and $d_n$ for the internal field,
respectively. The computation of these parameters has been the most challenging part in Mie computations due to the involvement of spherical Bessel functions up to
high order. With MATLAB’s built-in double-precision Bessel functions, the computation of the Mie coefficients has so far worked well up to size parameters exceeding 10,000; the coefficients are given in BH on p.100:
\begin{align}
	a_n&=\frac{m^2j_n(mx)[xj_n(x)]'-\mu_1j_n(x)[mxj_n(mx)]'}{m^2j_n(mx)[xh_n^{(1)}(x)]'-\mu_1h_n^{(1)}(x)[mxj_n(mx)]'}\\
	b_n&=\frac{\mu_1j_n(mx)[xj_n(x)]'-j_n(x)[mxj_n(mx)]'}{\mu_1j_n(mx)[xh_n^{(1)}(x)]'-h_n^{(1)}(x)[mxj_n(mx)]'}\\
	c_n&=\frac{\mu_1j_n(x)[xh_n^{(1)}(x)]'-\mu_1h_n^{(1)}(x)[xj_n(x)]'}{\mu_1j_n(mx)[xhh_n^{(1)}(x)]'-h_n^{(1)}(x)[mxj_n(mx)]'}\\
	d_n&=\frac{\mu_1mj_n(x)[xh_n^{(1)}(x)]'-\mu_1mh_n^{(1)}(x)[ xj_n(x)]'}{m j_n(mx)[xh_n^{(1)}(x)]'-\mu_1h_n^{(1)}(x)[mxj_n(mx)]'}
\end{align}
where m is the refractive index of the sphere relative to the ambient medium, $x=ka$ is the size parameter, $a$ the radius of the sphere and $k=2\pi/\lambda$ is the wave number and $\lambda$ the wavelength in the ambient medium. In deviation from BH, $\mu_1$ is the ratio of the magnetic permeability of the sphere to the magnetic permeability of the ambient medium (corresponding to $\mu_1/\mu$ in BH). The functions $j_n(z)$ and $h_n^{(1)}(z)=j_n(z)+iy_n(z)$ are spherical Bessel functions of order $n$ ($n=1,2,\ldots$) and of the given arguments, $z=x$ or $mx$, respectively, and primes mean derivatives with respect to the argument. The derivatives follow from the spherical Bessel functions themselves, namely
\begin{equation}
	\begin{split}
		[zj_n(z)]'&=zj_{n-1}(z)-nj_n(z)\\
		[zh_n^{(1)}(z)]'&=zh_{n-1}^{(1)}(z)-nh_n^{(1)}(z)
	\end{split}
\end{equation}
For completeness, the following relationships between Bessel and spherical Bessel functions are given:
\begin{align}
	j_n(z)&=\sqrt{\frac{\pi}{2z}}J_{n+0.5}(z)\\
	y_n(z)&=\sqrt{\frac{\pi}{2z}}Y_{n+0.5}(z)
\end{align}
Here, $J_\nu$ and $Y_\nu$ are Bessel functions of the first and second kind. For $n=0$ and $1$ the spherical Bessel functions are given (BH, p. 87) by
\begin{equation}
	\begin{split}
		j0(z)&=\sin{z}/z\\
		j1(z)&=\sin{z}/z^2-\cos{z}/z\\
		y0(z)&=-\cos{z}/z\\
		y1(z)&=-\cos{z}/z^2-\sin{z}/z
	\end{split}
\end{equation}
and the recurrence formula
\begin{equation}
	f_{n-1}(z)+f_{n+1}(z)=\frac{2n+1}{z}f_n(z)
\end{equation}
where $f_n$ is any of the functions $j_n$ and $y_n$. Taylor-series expansions for small arguments of $j_n$ and $y_n$ are given on p. 130 of BH. The spherical Hankel functions are linear combinations of $j_n$ and $y_n$. Here, the first type is required
\begin{equation}
	h_n^{(1)}(z)=j_n(z)+iy_n(z)
\end{equation}
The following related functions are also used in Mie theory (although we try to avoid them here):
\begin{equation}
	\begin{split}
		\psi_n(z)=zj_n(z)\\
		\chi_n(z)=-zy_n(z)\\
		\xi_n(z)=zh_n^{(1)}(z)
	\end{split}
\end{equation}
Often $\mu_1=1$; then, (4.52-4.53) simplify to
\begin{align}
	a_n&=\frac{m^2j_n(mx)[xj_n(x)]'-j_n(x)[mxj_n(mx)]'}{m^2j_n(mx)[xh_n^{(1)}(x)]'-h_n^{(1)}(x)[mxj_n(mx)]'}\\
	b_n&=\frac{j_n(mx)[xj_n(x)]'-j_n(x)[mxj_n(mx)]'}{j_n(mx)[xh_n^{(1)}( x)]'-h_n^{(1)}(x)[mxj_n(mx)]'}\\
	c_n&=\frac{j_n(x)[xh_n^{(1)}(x)]'-h_n^{(1)}(x)[xj_n(x)]'}{j_n(mx)[xh_n^{(1)}(x)]'-h_n^{(1)}(x)[mxj_n(mx)]'}\\
	d_n&=\frac{mj_n(x)[xh_n^{(1)}(x)]'-mh_n^{(1)}(x)[xj_n(x)]'}{m^2j_n(mx)[xh_n^{(1)}(x)]'-h_n^{(1)}(x)[mxj_n(mx)]'}
\end{align}
The parameters used in radiative transfer depend on $a_n$ and $b_n$, but not on $c_n$ and $d_n$. The latter coefficients are needed when the electric field inside the sphere is of interest, e.g. to test the field penetration in the sphere, to study the distribution of heat sources or to compute absorption. The absorption efficiency $Q_{abs}$, however, can also be computed from the scattered radiation, Equations (3.25), (4.61-62) to be shown below.

\end{document}

